\newcommand{\bvec}[1]{\boldsymbol{#1}}
\newcommand\blfootnote[1]{%
  \begingroup
  \renewcommand\thefootnote{}\footnote{#1}%
  \addtocounter{footnote}{-1}%
  \endgroup
}


\documentclass{article}
\usepackage[utf8]{inputenc}

\title{stein-lab-rot}
\author{Nikolai }
\date{May 2020}

\usepackage{natbib}
\usepackage{graphicx}

% \usepackage{hyperref}
% \usepackage{xcolor}
% \usepackage{caption}
% \usepackage{subcaption}
\usepackage{amsmath}
\usepackage{amssymb}
% \usepackage{wrapfig}
% \usepackage{url}
% \usepackage{caption}
% \usepackage{subcaption}
% \usepackage{amsthm}

\begin{document}

\maketitle

\section{SVGD}
    The Stein operator is defined as
    \begin{equation}
        \label{eq:stein_operator}
        \mathcal{A}_p \phi(x) = \phi(x) \nabla \log p(x) + \nabla \phi(x)
    \end{equation}
    Stein discrepancy is defined by

    \begin{equation}
        \label{eq:stein_discrepancy}
        \mathbb{D}( q, p ) = \max_{ \phi \in \mathcal{H} } 
        \left\{ 
          \mathbb{E}_{ x \sim q } 
            \left[ 
                \text{trace} ( A_p \phi(x) ) 
            \right]^2
            \ | \ ||\phi|| \leq 1 
        \right\}
    \end{equation}

    The maximising $\phi$ for the Stein discrepancy can found analytically if $\mathcal{H}$ is a RKHS. It is given by
    \begin{equation}
        \label{eq:phi_max}
        \phi_{ q, p } (x) = \mathbb{E}_{ y \sim q } \left[ \mathcal{A}_p k(x, y) \right]
    \end{equation}
    where $k(x,y)$ is the reproducing kernel.
    The central theorem of SVGD is
    \begin{equation}
        \label{eq:svgd}
        \frac{d}{dt}\ KL( q^t \| p ) = - \mathbb{E} \left[ \text{trace} \left( \mathcal{A}_p \phi(x) \right) \right]
    \end{equation}
    All the above is from \cite{svgd}


\section{Thermodynamic Integration and SVGD}
    Opper reasoned this should work for SVGD by analogy to the following, which I think is a more standard gradient
    flow transportation. Assume that we have a time derivative of the KL divergence that can be written as a gradient.
    \begin{equation}
        \frac{d}{dt}\ KL( q^t \| p ) = - \nabla F_t.
    \end{equation}
    Where $q(x)$ is a variational distribution trying to approximate $p(x)$ the true posterior distributions
    Assuming that this converges in the sense that the final distribution $q^{\infty}$ has 0 KL divergence with 
    $p(x)$ we can write
    \begin{align}
    \label{eq:therm_int}
    0 &= \text{KL}( q^{\infty} \| p ) \\
    &= \text{KL} ( q^0 \| p ) - \int_0^{\infty} \frac{d}{dt} \text{KL} ( q^t \| p ) dt \\
    &= \text{KL}( q^0 \| p ) + \int_0^{\infty} \nabla F_t dt
    \end{align}

    The second ingredient to finding $\log Z$ from this is the assumption that $p(x) = \frac{ e^{-V(x)} }{ Z }$, 
    where $Z = \int e^{-V(x, y)} dx$ is the evidence. In this case we can write
    \begin{align}
    \label{eq:gibbs_kl}
        \text{KL}( q^t \| p ) &= \int_X q^t(x) \log \frac{ q^t(x) }{ p(x) } dx \\
                        &= \int_X q^t(x) \log q^t(x) - q^t(x) \log \frac{ e^{-V(x, y)} }{ Z } dx \\
                        &= \int_X q^t(x) \log q^t(x) dx + \int_X q^t(x) V(x, y) dx + \log Z \int_X q^t(x) dx \\
                        &= \int_X q^t(x) \log q^t(x) dx + \int_X q^t(x) V(x, y) dx + \log Z
    \end{align}

    Now the first term in (\ref{eq:therm_int}) is (\ref{eq:gibbs_kl}) at $t=0$ and the integrand in the second term is
    what is calculated at every step in the original algorithm, so it the integral can be estimated, allowing us to 
    recover $\log Z$. SVGD does not actually calculate the full time derivative at every step, but the theory does 
    include an analytical expression for it so this method should work with SVGD as well.

    This gives us $\log Z$ as :
    \begin{align}
    \label{eq:log_z}
        \log Z = - \int_X q^t(x) [\log q^t(x) dx +  V(x, y)] dx 
            - \int_0^\infty \nabla F_t dt
    \end{align}


    \newpage

\subsection{Computing Thermodynamic integration}
\subsubsection{Estimating the Gradient for SVGD}
    The time derivative of the KL divergence is not a true gradient nonetheless we should be able to use
    (\ref{eq:svgd}) in combination with (\ref{eq:phi_max}) to approximate the second term in (\ref{eq:log_z}).
    Combining the two the derivate of the KL divergence, expressed in terms of the kernel of the RKHS, is:
    \begin{align}
        \label{eq:F_approximation}
        \frac{d}{dt}\ KL( q^t \| p ) = -& \frac{1}{n}\sum_{x_i,x_j}^n  k(x_i,x_j) \nabla_{x_i} \log p(x_i)^\top \nabla_{x_i} \log p(x_i)  \\ 
        -& \frac{1}{n}\sum_{x_i,x_j}^n  \nabla_{x_i} k(x_i,x_j)^\top \nabla_{x_j} \log p(x_j)  \\ 
        -& \frac{1}{n}\sum_{x_i,x_j}^n  \nabla_{x_j} k(x_i,x_j)^\top \nabla_{x_i} \log p(x_i)  \\ 
        -& \frac{1}{n}\sum_{x_i,x_j}^n \left( \sum_k \frac{ \partial^2 }{ \partial (x_i)_k \partial (x_j)_k } 
            k(x_i,x_j) \right)
    \end{align}

    When using an RBF kernel we can use the additional simplifications reducing the approximation to

    \begin{align}
        \frac{d}{dt}\ KL( q^t \| p ) \approx -& \frac{1}{n}\sum_{x_i,x_j}^n  \Bigg[
            k(x_i,x_j) \Bigg( \nabla_{x_i} V(x_i)^\top \nabla_{x_i} V(x_i)  
            + \frac{2}{h}\Big( d - \frac{2}{h} \|x_i - x_j\|^2 \Big) \Bigg)
        \Bigg]
    \end{align} 
    % In order to simplify the implementation/optimize it we can rewrite this as
    % % In order to simplify the implementation we can write this in terms of matrix multiplications,
    % % let 
    % % $$K_{ij} = k(x_i,x_j) $$
    % be the matrix of kernel pairwise evaluations on the particles,
    % $$GK(x_i)_{\cdot j} = \nabla_{1} k(x_i,x_j)$$
    % % what is a good way to indicate rows of columns of a matrix?
    % be the matrix whose columns are the gradients of the kernel (with respect 
    % to the first argument) evaluated at $(x_i, x_j)$ (so the columns of $GK(x_i)$ are the gradients with first argument
    % held fixed at $x_i$ and second argument $j$=column number, and let 
    % $$GLP_{\cdot j} = \nabla \log p(x_j)$$
    % the matrix whose
    % columns are the gradient of the log probability evaluated at each particle, then

    % % \begin{align}
    % %     \label{eq:F_approx_matrix}
    % %     -& \frac{1}{n}  \sum_{x_i}^n \nabla \log p(x_i)^\top K \nabla \log p(x_i)  \\ 
    % %     -& 2 \frac{1}{n}\sum_{x_i}^n  GK(x_i) GLP  \\
    % %     +& \frac{1}{n}\sum_{x_i,x_j}^n \left( \sum_k \frac{ \partial^2 }{ \partial (x_i)_k \partial (x_j)_k } 
    % %         k(x_i,x_j) \right)
    % % \end{align}

    % \begin{align}
    %     \label{eq:F_approx_matrix}
    %     - \frac{1}{n}  \Bigg( &\sum_{i}^n \| \nabla \log p(x_i) \|^2 \sum_{x_j}^{n}k(x_i,x_j) \\ 
    %     -& 2 \sum_{i}^n  GK(x_i) GLP  \\
    %     +& \sum_{i}^n \Big( \sum_k \frac{ \partial^2 }{ \partial (x_i)_k \partial (x_i)_k } k(x_i,x_i) \Big) \\ 
    %     +& 2\sum_{i<j}^n \Big( \sum_k \frac{ \partial^2 }{ \partial (x_i)_k \partial (x_j)_k } 
    %         k(x_i,x_j) \Big)\Bigg)
    % \end{align}

\subsection{Experiments}
    Possible targets to try it on:
    \begin{itemize}
        \item Gaussians
        \item neural net
        \item augmented regression (i.e. with polynomial regressors)
    \end{itemize}


\section{Appendix}
\subsection{A: Derivation of gradient estimation}
    This appendix contains the calculation to get from the derivative of the KL-divergence in (\ref{eq:svgd})
    to the update for the algorithm in (\ref{eq:F_approximation}).
    Starting with the original theorem and expanding the Stein operator we get
    \begin{align}
        \frac{d}{dt}\ KL( q^t \| p ) &= - \mathbb{E}_{x \sim q} \left[ \text{trace} \left( \mathcal{A}_q \phi(x) \right) \right] \\
             &=  - \mathbb{E}_{x \sim q} \left[ 
                 \text{trace} \left( \phi(x) \nabla_x \log p(x)^\top  + \nabla_x \phi(x) \right)
             \right]\
    \end{align}

    Substituting (\ref{eq:phi_max}) for $\phi$ and the second Stein operator

    \begin{gather}
             =  - \mathbb{E}_{x \sim q} \left[ 
                     \text{trace} \left( \mathbb{E}_{y \sim q} \left[ \mathcal{A}_p k(y,x) \right] \nabla_x \log p(x)^\top
                            + \nabla_x \mathbb{E}_{y \sim q } \left[ \mathcal{A}_p k(y, x) \right] \right) 
                    \right] \\
            =  - \mathbb{E}_{x \sim q} \Big[ 
                         \text{trace}  \Big(
                          \mathbb{E}_{y\sim q} \left[ k(y,x) \nabla_y \log p(y)^\top 
                          + \nabla_{y} k(y,x) \right] \nabla_x \log p(x)^\top \\
                   + \nabla_x \mathbb{E}_{y \sim q } \left[ 
                              k(y,x) \nabla_y \log p(x)^\top 
                              + \nabla_{y} k(y,x)
                          \right] 
                  \Big) \Big] \\
            = - \mathbb{E}_{x \sim q} \Big[ 
                 \text{trace} \Big( \mathbb{E}_{y \sim q} [ k(y,x) \nabla_y \log p(y)^\top \nabla_x \log p(x)^\top \Big) \\
                 + \text{trace} \Big( \mathbb{E}_{y \sim q} [ \nabla_{y}  k(y,x) ] \nabla_x \log p(x)^\top \Big) \\
                 + \text{trace} \Big( \nabla_x \mathbb{E}_{y \sim q} [ k(y,x) \nabla_y  \log p(y)^\top ] \Big) \\
                 + \text{trace} \Big( \nabla_x \mathbb{E}_{y \sim q} [ \nabla_y  k(y,x)] \Big)
                 \Big] \\
             = - \mathbb{E}_{x \sim q} \mathbb{E}_{y \sim q} \Big[
                 \text{trace} \Big( k(y,x) \nabla_y \log p(y) \nabla_x \log p(x)^\top \Big) \\
                 + \text{trace} \Big( \nabla_y  k(y,x) \nabla_x \log p(x)^\top \Big) \\
                 + \text{trace} \Big( \nabla_x k(y,x) \nabla_y  \log p(y)^\top \Big) \\
                 + \text{trace} \Big( \nabla_x \nabla_y  k(y,x) \Big) 
    \end{gather}

    The trace of an outer product is simply the inner product of the same vectors so the above can 
    be further simplified to

    \begin{align}
        \frac{d}{dt} \text{KL}( q^t \| p ) = - \mathbb{E}_{x\sim q} \mathbb{E}_{y \sim q} \big[
            & k(x,y) (\nabla_y \log p(y))^\top \nabla_x \log p(x) \\
            &+ (\nabla_y k(x,y) )^\top \nabla_x \log p(x) \\
            &+ (\nabla_x k(x,y) )^\top \nabla_y \log p(y) \\
            &+ \sum_i \frac{ \partial^2 }{ \partial x_i \partial y_i } k(x,y) \big]
    \end{align}
    \begin{align}
        \frac{d}{dt} \text{KL} ( q^t \| p ) =
        -& \mathbb{E}_{x,y\sim q} \left[ k(x,y) \nabla_y \log p(y)^\top \nabla \log p(x) \right] \\ 
        -& \mathbb{E}_{x,y\sim q} \left[ \nabla_x k(x,y)^\top \nabla_y \log p(y) \right] \\ 
        -& \mathbb{E}_{x,y\sim q} \left[ \nabla_y k(x,y^\top \nabla_x \log p(x) \right] \\ 
        -& \mathbb{E}_{x,y\sim q} \left[ \sum_i \frac{ \partial^2 }{ \partial x_i \partial y_i } k(x,y) \right]
    \end{align}

    Now $\nabla_x k(y,x) = - \nabla_y k(y,x)$ for stationary kernels, and since we add the trace and the expectation of 
    the second and third terms, both of which are linear functions, they cancel.
    When using the RBF kernel we can also simplify the last term because in that case $\nabla_x k(x,y) = -\frac{2}{h} (x-y) k(x,y) $,
    or $\partial_{x_i} k(x,y) = -\frac{2}{h} (x_i - y_i) k(x,y)$. Using this 
    \begin{align}
        &\text{trace} \Big( \nabla_x \nabla_y k(x,y) \Big) \\
        = &\text{trace} \Big( \nabla_x \big( \frac{2}{h} (x-y)k(x,y) \big) \Big) \\
        = &\frac{2}{h} \sum_{i}^{n} k(x,y) + (x_i - y_i) \partial_{x_i} k(x,y) \\
        = & \frac{2k(x,y)}{h}\Big( d - \frac{2}{h}\|x-y\|^2 \Big)
    \end{align}
    where $d$ is the dimensionality of $x$
    reducing the derivative to
    \begin{align}
        \frac{d}{dt}\ KL( q^t \| p ) =\mathbb{E}_{x,y\sim q} \Bigg[ -& \frac{1}{n}\sum_{x_i,x_j}^n  k(x_i,x_j) \nabla_{x_i} \log p(x_i)^\top \nabla_{x_i} \log p(x_i)  \\ 
        -& \frac{2k(x,y)}{h}\Big( d - \frac{2}{h}\|x-y\|^2 \Big)\Bigg].
    \end{align}

    Approximating $q$ with the empirical distribution
    (Note that in this step the meaning of the indices changed. Before $x_i$ indicated the $i^{th}$
    coordinate of the vector $x$, below $x_i$ denotes the $i^{th}$ particle and $(x_i)_k$ denotes
    its $k^{th}$ coordinate.
    For arbitrary kernels we get
    \begin{align}
        \frac{d}{dt}\ KL( q^t \| p ) \approx -& \frac{1}{n}\sum_{x_i,x_j}^n  k(x_i,x_j) \nabla_{x_i} V(x_i)^\top \nabla_{x_i} V(x_i)  \\ 
        -& \frac{1}{n}\sum_{x_i,x_j}^n  \nabla_{x_i} k(x_i,x_j)^\top \nabla_{x_j} V(x_j)  \\ 
        -& \frac{1}{n}\sum_{x_i,x_j}^n  \nabla_{x_j} k(x_i,x_j)^\top \nabla_{x_i} V(x_i)  \\ 
        -& \frac{1}{n}\sum_{x_i,x_j}^n \left( \sum_k \frac{ \partial^2 }{ \partial (x_i)_k \partial (x_j)_k } 
            k(x_i,x_j) \right)
    \end{align}
    which simplifies to 
    \begin{align}
        \frac{d}{dt}\ KL( q^t \| p ) \approx -& \frac{1}{n}\sum_{x_i,x_j}^n  k(x_i,x_j) \nabla_{x_i} V(x_i)^\top \nabla_{x_i} V(x_i)  \\ 
        -& \frac{1}{n}\sum_{x_i,x_j}^n \frac{2k(x_i, x_j)}{h}\Big( d - \frac{2}{h} \|x_i - x_j\|^2 \Big)
    \end{align} 

    % not sure if the stuff below will be needed
    Inserting $p(x) = \frac{e^{-V(x)}}{Z}$ we would get
    \begin{align}
        \frac{d}{dt} \text{KL}( q^t \| p ) =
        -& \mathbb{E}_{x,y\sim q} \left[ k(x,y) \nabla_y V(x)^\top \nabla_x V(x) \right] \\ 
        -& \mathbb{E}_{x,y\sim q} \left[ \nabla_x k(x,y)^\top \nabla_y V(y) \right] \\ 
        -& \mathbb{E}_{x,y\sim q} \left[ \nabla_x k(x,y)^\top \nabla_x V(x) \right] \\ 
        -& \mathbb{E}_{x,y\sim q} \left[ \sum_i \frac{ \partial^2 }{ \partial x_i \partial y_i } k(x,y) \right]
        .
    \end{align}

    The SVGD update written in terms of $V(x)$ is $x_i^{l+1} = x_i^l + \epsilon_l \phi(x_n^l)$
    \begin{align}
        \phi(x) = \frac{1}{n} \sum_{i=0}^{n} \left[
            - k(x_i^l, x) \nabla V(x_i^l) + \nabla k(x_i^l, x)
        \right]
    \end{align}
    where the gradient in the second term is taken with respect to the first variable.
    \end{document}
