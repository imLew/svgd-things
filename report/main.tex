\newcommand{\bvec}[1]{\boldsymbol{#1}}
\newcommand{\tr}{\textrm{trace}}
\newcommand{\KL}{\textrm{KL}}
\newcommand{\A}{\mathcal{A}}
\newcommand{\F}{\mathcal{F}}
\newcommand{\E}{\mathbb{E}}
\newcommand{\D}{\mathbb{D}}
\newcommand{\R}{\mathbb{R}}
\newcommand\blfootnote[1]{%
  \begingroup
  \renewcommand\thefootnote{}\footnote{#1}%
  \addtocounter{footnote}{-1}%
  \endgroup
}


\documentclass{article}
\usepackage[utf8]{inputenc}

\title{stein-lab-rot}
\author{Nikolai }
\date{May 2020}

\usepackage{natbib}
\usepackage{graphicx}

% \usepackage{hyperref}
% \usepackage{xcolor}
% \usepackage{caption}
% \usepackage{subcaption}
\usepackage{amsmath}
\usepackage{amssymb}
% \usepackage{wrapfig}
% \usepackage{url}
% \usepackage{caption}
% \usepackage{subcaption}
% \usepackage{amsthm}

\begin{document}

\maketitle

\section{SVGD}
The Stein operator is defined as
\begin{equation}
    \label{eq:stein_operator}
    \A_p \phi(x) = \phi(x) \nabla \log p(x) + \nabla \phi(x)
\end{equation}
Stein discrepancy is defined by
\begin{equation}
    \label{eq:stein_discrepancy}
    \D( q, p ) = \max_{ \phi \in \mathcal{H} } 
    \left\{ 
        \E_{ x \sim q } 
        \left[ 
            \tr ( A_p \phi(x) ) 
        \right]^2
        \ | \ ||\phi|| \leq 1 
    \right\}
\end{equation}
The maximising $\phi$ for the Stein discrepancy can found analytically if $\mathcal{H}$ is a RKHS. It is given by
\begin{equation}
    \label{eq:phi_max}
    \phi_{ q, p } (x) = \E_{ y \sim q } \left[ \A_p k(x, y) \right]
\end{equation}
where $k(x,y)$ is the reproducing kernel.
The central theorem of SVGD is
\begin{equation}
    \label{eq:svgd}
    \frac{d}{dt}\ KL( q^t \| p ) = - \E_{q(x)} \left[ \tr \left( \A_p \phi(x) \right) \right]
\end{equation}
All the above is from \cite{svgd}


\section{Thermodynamic Integration and SVGD}
Opper reasoned this should work for SVGD by analogy to the following, which I think is a more standard gradient
flow transportation. Assume that we have a time derivative of the KL divergence that can be written as a gradient.
\begin{equation}
    \frac{d}{dt}\ KL( q^t \| p ) = - \nabla F_t.
\end{equation}
Where $q(x)$ is a variational distribution trying to approximate $p(x)$ the true posterior distributions
Assuming that this converges in the sense that the final distribution $q^{\infty}$ has 0 KL divergence with 
$p(x)$ we can write
\begin{align}
\label{eq:therm_int}
0 &= \KL( q^{\infty} \| p ) \\
&= \KL( q^0 \| p ) - \int_0^{\infty} \frac{d}{dt} \KL( q^t \| p ) dt \\
&= \KL( q^0 \| p ) + \int_0^{\infty} \nabla F_t dt
\end{align}

The second ingredient to finding $\log Z$ from this is the assumption that $p(x) = \frac{ e^{-V(x)} }{ Z }$, 
where $Z = \int e^{-V(x, y)} dx$ is the evidence. In this case we can write
\begin{align}
\label{eq:gibbs_kl}
    \KL( q^t \| p ) &= \int_X q^t(x) \log \frac{ q^t(x) }{ p(x) } dx \\
                    &= \int_X q^t(x) \log q^t(x) - q^t(x) \log \frac{ e^{-V(x, y)} }{ Z } dx \\
                    &= \int_X q^t(x) \log q^t(x) dx + \int_X q^t(x) V(x, y) dx + \log Z \int_X q^t(x) dx \\
                    &= \int_X q^t(x) \log q^t(x) dx + \int_X q^t(x) V(x, y) dx + \log Z
\end{align}

Now the first term in (\ref{eq:therm_int}) is (\ref{eq:gibbs_kl}) at $t=0$ and the integrand in the second term is
what is calculated at every step in the original algorithm, so it the integral can be estimated, allowing us to 
recover $\log Z$. SVGD does not actually calculate the full time derivative at every step, but the theory does 
include an analytical expression for it so this method should work with SVGD as well.

This gives us $\log Z$ as :
\begin{align}
\label{eq:log_z}
    \log Z = - \int_X q^t(x) [\log q^t(x) dx +  V(x, y)] dx 
        - \int_0^\infty \nabla F_t dt
\end{align}


\subsection{Estimating the Gradient for SVGD}
The time derivative of the KL divergence is not a true gradient nonetheless we should be able to use
(\ref{eq:svgd}) in combination with (\ref{eq:phi_max}) to approximate the second term in (\ref{eq:log_z}).
Combining the two the derivate of the KL divergence, expressed in terms of the kernel of the RKHS, is:


\section{Appendix}
Derivation of gradient estimation
\begin{align}
    \frac{d}{dt} \KL( q^t \| p ) = - \E_{x\sim q} \E_{y \sim q} \big[
        & k(x,y) (\nabla_y \log p(y))^\top \nabla_x \log p(x) \\
        &+ (\nabla_y k(x,y) )^\top \nabla_x \log p(x) \\
        &+ (\nabla_x k(x,y) )^\top \nabla_y \log p(y) \\
        &+ \sum_i \frac{ \partial^2 }{ \partial x_i \partial y_i } k(x,y) \big]
\end{align}
\begin{align}
    \frac{d}{dt} \KL( q^t \| p ) =
    -& \E_{x,y\sim q} \left[ k(x,y) (\nabla_y \log p(y))^\top \nabla_x \log p(x) \right] \\ 
    +& 2 \E_{x,y\sim q} \left[ (\nabla_x k(x,y) )^\top \nabla_y \log p(y) \right] \\ 
    +& \E_{x,y\sim q} \left[ \sum_i \frac{ \partial^2 }{ \partial x_i \partial y_i } k(x,y) \right]
\end{align}

Inserting $p(x) = \frac{e^{-V(x)}}{Z}$ we get
\begin{align}
    \frac{d}{dt} \KL( q^t \| p ) =
    -& \E_{x,y\sim q} \left[ k(x,y) (\nabla_y V(x))^\top \nabla_x V(x) \right] \\ 
    +& 2 \E_{x,y\sim q} \left[ (\nabla_x k(x,y) )^\top \nabla_y V(y) \right] \\ 
    +& \E_{x,y\sim q} \left[ \sum_i \frac{ \partial^2 }{ \partial x_i \partial y_i } k(x,y) \right]
    .
\end{align}
Approximating $q$ with the empirical distribution
\begin{align}
    -& \frac{1}{n}\sum_{x_i,x_j} \left[ k(x_i,x_j) (\nabla V(x_i))^\top \nabla V(x_i) \right] \\ 
    -& 2 \frac{1}{n}\sum_{x_i,x_j} \left[ (\nabla_x k(x_i,x_j) )^\top \nabla V(x_j) \right] \\ 
    +& \frac{1}{n}\sum_{x_i,x_j} \left[ \sum_i \frac{ \partial^2 }{ \partial x_i \partial y_i } k(x_i,x_j) \right]
\end{align}

The SVGD update written in terms of $V(x)$ is $x_i^{l+1} = x_i^l + \epsilon_l \phi(x_n^l)$
\begin{align}
    \phi(x) = \frac{1}{n} \sum_{i=0}^{n} \left[
        - k(x_i^l, x) \nabla V(x_i^l) + \nabla k(x_i^l, x)
    \right]
\end{align}
where the gradient in the second term is taken with respect to the first variable.
\end{document}
